% Created 2017-03-20 Mon 14:32
% Intended LaTeX compiler: xelatex
\documentclass[a4paper, notitlepage]{report}
\usepackage{graphicx}
\usepackage{grffile}
\usepackage{longtable}
\usepackage{wrapfig}
\usepackage{rotating}
\usepackage[normalem]{ulem}
\usepackage{amsmath}
\usepackage{textcomp}
\usepackage{amssymb}
\usepackage{capt-of}
\usepackage{hyperref}
% set the margins here, if they need to be modified
\usepackage[a4paper, left=1.3in, right=1.0in]{geometry}

\usepackage[style=ieee,backend=biber]{biblatex}

% The linespacing you want
\linespread{1.4}

% Lets you pick fonts. Only works if you compile with xelatex or luatex
\usepackage[no-math]{fontspec}

\usepackage{unicode-math}

\usepackage{stackengine}
\stackMath

% Used to make the captions on figures sans serif
\usepackage{caption}

% For including images
\usepackage{graphicx}
\usepackage{float}

\usepackage{color}

% For fancy code listings
\usepackage{minted}

% For tables that span multiple pages elegantly
\usepackage{longtable}

% For maintaining a list of abbreviations
\usepackage{nomencl}

% For drawing diagrams
\usepackage{tikz}

% These closely match the SCSS word template fonts
% Won't work unless you compile with xelatex or luatex
\setmainfont[Mapping=tex-text]{Times New Roman}
\setsansfont{Helvetica}
\setmonofont[Scale=MatchLowercase]{Monaco}
\setmathfont{TeX Gyre Termes Math}

\hypersetup{
  colorlinks,
  citecolor=black,
  filecolor=black,
  linkcolor=black,
  urlcolor=black
}

% Change these as appropriate, and they'll be filled in automatically on the
% cover page. You can also use them throughout the document, so as not to have
% to type them again all the time.
\newcommand \authorname{Eoin Houlihan}
\newcommand \authoremail{ehoulih@tcd.ie}
\newcommand \supervisorname{Dr. Glenn Strong}
\newcommand \degreetitle{B.A.(Mod.) Computer Science}
\newcommand \projecttitle{Dependent Types in Practice}

% All the commands are defined in this file
\newcommand \inserttitlepage{\include{tcd/titlepage}}
\newcommand \insertabstract{\begin{abstract}
\thispagestyle{plain}
Major software vulnerabilities in critical software has led to more interest in
formal verification of our software. One way that has presented itself to verify
our code is by using dependent types, an idea from the 1970s present in a number of
upcoming programming languages.

We present a study into the current state-of-the-art dependently-typed
programming languages. Idris was chosen as the tool to carry out this study. The
methodology involved carrying out case studies into applying dependent types to
real-world programming problems. An evaluation of the outcomes and the current
landscape of dependently-typed programming languages applied in a practical
sense is provided as a result of carrying out these case studies.
\end{abstract}
}
\newcommand \declaration{\include{tcd/declaration}}
\newcommand \acknowledgements{\chapter*{Acknowledgements}
I'd firstly like to thank my parents for giving me the opportunity, support and
encouragement I needed throughout my degree.

I'd like to thank Glenn Strong for his continued guidance and suggestions
throughout this project as my supervisor.

I'd also like to thank the budding Idris and dependent type community,
particularly those in the \#idris IRC channel, for being patient enough to
answer any questions that I had.
}
\newcommand \permissiontolend{\include{tcd/permissiontolend}}
\newcommand \ibidp{(\emph{Ibid.})}
\newcommand \ibid{\emph{Ibid.}}

\newcommand{\argmax}[1]{\underset{#1}{\operatorname{arg}\,\operatorname{max}}\;}

\usepackage{datetime}

% Modify this so that the submission date on the signatures is correct
\def\drawsig#1{
    \vbox{\hsize=8cm
    \kern 2cm \hrule \kern 1ex
    \hbox to \hsize{\strut\hfil #1, Friday 5$^{\mathrm{th}}$ of May, 2017 \hfil}}
}


% This draws the lines that you sign on for the declaration pages
\usetikzlibrary{decorations.markings}
\newcommand{\signedanddate} [3] []{
    \begin{center}
        \begin{tikzpicture}
            \draw[#2-#3, thick, #1] (0,0) to (0.5\linewidth,0);
            \node[align=left, below] at (1.1,0) {\authorname};
            \draw[#2-#3, thick, #1] (0.7\linewidth,0) to (\linewidth,0);
            \node[align=left,below] at(0.729\linewidth,0) {Date};
        \end{tikzpicture}
    \end{center}
}

\newcommand \signed[1]{
\signedanddate{}{}

}

\newcommand \needcite[1]{\underline{#1}}

\newcommand \abbrev[2]{#1\nomenclature{#1}{#2}}


% For changing the names of the List of Listings, etc.
\renewcommand*{\contentsname}{Table of Contents}
\renewcommand*{\nomname}{Abbreviations}

% Make the captions sans serif
\renewcommand{\captionfont}{\sffamily}

\usepackage{draftwatermark}
\SetWatermarkLightness{0.95}
\author{Eoin Houlihan}
\date{\today}
\title{}
\hypersetup{
 pdfauthor={Eoin Houlihan},
 pdftitle={},
 pdfkeywords={},
 pdfsubject={},
 pdfcreator={Emacs 25.2.1 (Org mode 9.0.5)},
 pdflang={English}}
\begin{document}

\inserttitlepage
\pagenumbering{roman}
\declaration
\permissiontolend
\insertabstract
\acknowledgements
\tableofcontents
\newpage
\pagenumbering{arabic}

\part{Introduction and Background}
\label{sec:org893bb68}
\chapter{Introduction}
\label{sec:org996e7bc}
\chapter{Background}
\label{sec:org339e84f}
This chapter aims to give an understanding of the background and necessary
concepts that will be used throughout the report. A basic understanding of
functional programming ideas such as algebraic data types, recursion and the
fundamentals of the Hindley-Milner type system with respect to languages such as
Haskell is assumed of the reader. It also gives a broad overview of the current
state of the art dependently-typed programming languages with a more in-depth
look at Idris in particular.

\section{Intuitionistic Type Theory}
\label{sec:org7763afc}
Intuitionistic type theory is a type theory based on mathematical
constructivism. Constructive mathematics is an alternative foundational theory
of mathematics that argues that construction of a mathematical object is
necessary to proving that such an object exists. Of particular note, the
intuitionistic logic which much of constructivism uses deviates from classical
logic systems in that proof by contradiction is not used and the law of the
excluded middle is not assumed as an axiom of the logic in general.

Drawing upon these ideas, Per Martin-Löf, a Swedish logician, developed a number
of successive type theories in the 1970s. This intuitionistic type theory (also
commonly referred to as Martin-Löf type theory) introduces a number of
interesting concepts. Most notable in terms of their influences on programming
language design were the concepts of \(\Pi\)-types and \(\Sigma\)-types. These
constructs can be seen as analogous to the logical quantifiers "forall" and
"exists" respectively. These concepts have served as the underpinning of the
development of dependently-typed programming languages and theorem provers based
on Martin-Löf type theory.

\section{Curry-Howard Isomorphism}
\label{sec:orgb9f2ab4}
From a modern computer science perspective it's almost taken for granted that
computability theory and mathematical proofs are inherently linked. For example,
many parallels can be drawn between the proof of Turing's Halting Problem and
Gödel's incompleteness theorems. Between the 1930s and the 1960s Haskell Curry
and William Alvin Howard began to formalise this direct link between computer
programs and mathematical proofs which is known as the Curry-Howard isomorphism.

According to the Curry-Howard isomorphism the type of an expression is
equivalent to a proposition of a logical formula. A term inhabiting that type is
therefore equivalent to a proof that the proposition holds. Some concrete value
exists that bears witness to the type being inhabited. In other words, a proof
can be constructed. This very much aligns with the constructivist view of
mathematics. Other correspondences can be shown such as between logical
implication and function types, conjunction and product types and between false
formulas and the uninhabited type, bottom (\(\bot\)). We can even see from the
shape of the syntax rules of both natural deduction and simply-typed lambda
calculus that these kinds of correspondences exist. As an example, the
relationship between the Modus Ponens rule and the function application rule.
\[ \scalebox{2}{$\frac{\Gamma \vdash \alpha \rightarrow \beta \qquad \Gamma
\vdash \alpha}{\Gamma \vdash \beta} \rightarrow E \qquad \frac{\Gamma \vdash
t:\alpha \rightarrow \beta \qquad \Gamma \vdash u:\alpha}{\Gamma \vdash
t\;u:\beta}$} \]

One of the consequences of this relationship is the possibility of a unification
at a primitive level between mathematical logic and the foundations of
computation. In practical terms this relationship has influenced the work on
programming languages such as Coq and Idris that allow proofs to be written as
programs that can be formalised, verified and executed. This is interesting to
the practice of software engineering as it gives us the power to reason about
program correctness by translating a mathematical proof of an algorithm to a
computer program and having the machine type-check (proof-check) it.

\section{Traditional Hindley-Milner Type Systems}
\label{sec:org34808c1}
Standard Hindley-Milner-esque type systems such as the one found in Haskell
allow us to express some different dependencies between the types of terms and
terms themselves. For example, terms can depend on other terms such as in this
Haskell function definition.

\begin{minted}[frame=single,fontsize=\scriptsize,linenos,escapeinside=\#\#]{haskell}
plusOne x = x + 1
\end{minted}

Here we can see that the term \texttt{plusOne} has been defined with respect to the
terms \texttt{x}, which is its argument and \texttt{1}, an integer. Types can also depend on
other types as shown here.

\begin{minted}[frame=single,fontsize=\scriptsize,linenos,escapeinside=\#\#]{haskell}
data List a = Nil
            | Cons a (List a)
\end{minted}

In this Haskell data type definition, the type constructor \texttt{List} depends on the
type \texttt{a} provided to it. This allows polymorphism and lists of any type.
Finally, in the world of Haskell, terms can depend on types. This is apparent in
polymorphic functions such as the identity function.

\begin{minted}[frame=single,fontsize=\scriptsize,linenos,escapeinside=\#\#]{haskell}
id :: a -> a
id x = x
\end{minted}

\section{Dependent Type Systems}
\label{sec:org8042975}
Dependent type systems extend this system of dependencies by allowing types to
depend on terms. This leads to much greater expressivity power in the type
system. For example, in a dependently typed system we can express types such as
the type of pairs of natural numbers where the second number is greater than the
first.

If we take the view of the Curry-Howard isomorphism that types are propositions
and terms are witnesses to a proof of that proposition then we can see the
advantages of a more expressive type system. We can now encode much more
sophisticated propositions in the type system and if we can prove them (i.e.
construct a value that inhabits that type) then we can guarantee much more
interesting correctness properties about the code that we are writing. For this
reason, dependent types have seen much use in the areas of formal verification
of computer programs and formal computer encoding of mathematical objects and
proofs.

There are 3 main concepts taken from Martin-Löf type theory and implemented in
dependently-typed programming languages.

\subsection{\(\Pi\)-types}
\label{sec:org6e4d25e}
\(\Pi\)-types are the types of functions whose return types depend on one or more of
their arguments. In other words these functions map values from some domain to
some non-fixed codomain that is determined by the input. In this sense the
return type is said to be dependent upon the input.

If we have a representation of \(n\textrm{-tuples}\) of some type \(A\),
\(\operatorname{Vect}(A,n)\), then the \(\Pi\)-type \(\Pi_{(n \mathbin{:} {\mathbb N})}
\operatorname{Vect}(A,n)\) represents the type of functions that given some
natural number \(n\) return a tuple of size \(n\) of elements of type \(A\). That is
to say that the type of the value returned by these functions is determined by
the argument to the functions.

\subsection{\(\Sigma\)-types}
\label{sec:org691b2bb}
\(\Sigma\)-types, also known as dependent pair types, are a more generalised form of
Cartesian product that model pairs of values where the type of the second
element depends on the first element.

Again using the \(\operatorname{Vect}\) representation of \(n\textrm{-tuples}\) of
some type \(A\), the \(\Sigma\)-type \(\Sigma_{(n \mathbin{:} {\mathbb N})}
\operatorname{Vect}(A,n)\) represents a pair of a natural number \(n\) and a tuple
of length \(n\) of values of type \(A\).

This representation is similar to the Haskell \texttt{List} type however there is extra
information in that the type of the \(\Sigma\)-type \(\operatorname{Vect}\) also
carries around a witness to its length expressed as a natural number. We say
that \(\operatorname{Vect}\) is "indexed" by the type \(A\) as well as the value
\(n\).

Being able to index types by both types and terms in the language is a key
feature of dependently-typed programming languages. These languages eliminate
the distinction between types and terms. Types and terms are unified as
equivalent constructs.

\subsection{The Equality Type}
\label{sec:orgad89623}
The equality type \(=\) is a special type used to denote proofs of equality
between two values. If there is an inhabitant of the type \(a \mathrel{=} b\) then
\(a\) and \(b\) are considered to be equal. This proof allows \(b\) to be used
anywhere \(a\) would have been used. There is only one inhabitant of the type \(a
\mathrel{=} a\), the reflexive proof of equality.

\[ \scalebox{2}{$\operatorname{refl} \mathbin{:} \Pi_{(a \mathbin{:} A)} (a
\mathrel{=} a)$} \]

This type is particularly useful in dependently-typed programming in that it can
be used as a witness that two terms are equivalent and allows a substitution of
one term for another to take place. With it, we can begin to develop
constructions of basic proofs and axioms such as \(n \mathbin{:} {\mathbb N}, n
\mathbin{-} n \mathrel{=} 0\).

\section{State of The Art Dependently-Typed Programming Languages}
\label{sec:org8de6d94}
\subsection{Agda}
\label{sec:orgf9b88c2}
Originally developed in the late 1990s by Catarina Coquand and subsequently
rewritten by Ulf Norell in 2007, Agda is a dependently typed programming
language with support for features such as dependent pattern matching and
definition of inductive data types.

For example the inductive data type representing the Peano natural numbers can
be declared as follows in Agda.

\begin{minted}[frame=single,fontsize=\scriptsize,linenos,escapeinside=\#\#]{agda}
data #$\mathbb{N}$# : Set where
  zero : #$\mathbb{N}$#
  suc : #$\mathbb{N}$# → #$\mathbb{N}$#
\end{minted}

There are two cases to consider here. \texttt{zero} is the base case. \texttt{suc} (standing
for successor) takes a natural number and returns a new natural number. It
represents a natural number plus 1. We will see more definitions of inductive
types similar to this one throughout the later chapters.

\subsection{Coq}
\label{sec:org2ac3edb}
Developed initially in the late 1980s at INRIA in France, Coq approaches
dependently-typed programming more from the mathematical side as an interactive
theorem prover. Coq is based on the Calculus of Constructions, a type theory
created by Thierry Coquand. Coq provides useful facilities for defining
inductive data types and includes a tactics language for doing interactive
proofs.

Notable work created using Coq includes the formally verified C compiler
CompCert \cite{compcert}, as well as a formally verified proof of the Four-Colour
Theorem \cite{gonthier_formal_2008} for graph colouring.

\subsection{Idris}
\label{sec:orgd183a89}
Idris is the work primarily of Edwin Brady and others at the University of St.
Andrews in Scotland.

\section{The Idris Programming Language}
\label{sec:org99f1579}
\subsection{Similarities to Haskell}
\label{sec:org3f42584}

\subsection{Total Functional Programming}
\label{sec:orga668b3e}
\begin{minted}[frame=single,fontsize=\scriptsize,linenos,escapeinside=\#\#]{haskell}
head :: [a] -> a
head (x:_) = x
\end{minted}

\part{Implementation and Case Studies}
\label{sec:org33fa0d2}
\chapter{Project}
\label{sec:org212db82}
Example \cite{McKinna_2006}
\chapter{Case Studies}
\label{sec:org1e065f4}

\part{Assessments and Future Work}
\label{sec:orgec4544b}
\chapter{Assessments and Conclusions}
\label{sec:org7aff037}
\chapter{Future Work}
\label{sec:orgc1524de}

\bibliography{report}
\appendix
\end{document}